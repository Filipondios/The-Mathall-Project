\documentclass{article}
\usepackage[utf8]{inputenc}
\usepackage{parskip}
\usepackage{amsmath}
\usepackage{lmodern}

\title{\textbf{Mathall Matrices Documentation}}
\author{Filipondios}
\date{2022}

\begin{document}
\maketitle
\newpage

\textbf{Introduction}

In this document you will find the necessary documentation to understand the procedures and
thoughts in all Mathall operations with matrices.\\

You need to know that this isn't a math book and its not made by a mathematician, so the most
probable thing is that the explanations aren't the best or not exactly how a mathematician would
tell, because this document is made by a programmer just like every other out there.\\

\textbf{Greetings}

The only greetings i have to give is to my algebra book, that sometimes is unreadable but at least
opens its pages for me.

\newpage 

\section{Operations with Matrices}
\subsection{Matrices addition}

The very first function that was added to Mathall was the addition of two matrices. It was really
easy to perform, we only needed to add every element of each column and row with the other matrix.
In other words, the resultant element of the row \textit{i} and column \textit{j} will be:

\[ e_{i_j} = m1_{i_j} + m2_{i_j} \]

Being $m_1$ the element in the row \textit{i} and column \textit{j} of a matrix called \textit{m1}
and $m_2$ the element in the row \textit{i} and column \textit{j} of a matrix called \textit{m2}.

Let's make an example and add two simple matrices:

\[
\begin{pmatrix}
    1 & 2 & 3 \\
    4 & 5 & 6 \\
    7 & 8 & 9 \\
\end{pmatrix}
  + 
\begin{pmatrix}
    9 & 8 & 7 \\
    6 & 5 & 4 \\
    3 & 2 & 1 \\
\end{pmatrix}
=
\begin{pmatrix}
    10 & 10 & 10 \\
    10 & 10 & 10 \\
    10 & 10 & 10 \\
\end{pmatrix}
\]

In terms of code, the most efficient way to do this operations is to making the
addition of each element in both matrices row by row. In \textit{Java} code:\\

for(int i = 0; i \textless resultant.length; i++){ \{

\quad \quad \quad for(int j = 0; i \textless resultant.length; j++){ \{

\quad \quad \quad \quad \quad \quad resultant[i][j] = matrix1[i][j] + matrix2[i][j];

\quad \quad \quad \}

\} \\

\subsection{Matrices subtraction}

This is really the same as the matrix addition but instead of making the addition of every element,
the subtraction. Let's see an example:

\[
\begin{pmatrix}
    9 & 8 & 7 \\
    6 & 5 & 4 \\
    3 & 2 & 1 \\
\end{pmatrix}
  - 
\begin{pmatrix}
    8 & 8 & 7 \\
    6 & 4 & 4 \\
    3 & 2 & 0 \\
\end{pmatrix}
=
\begin{pmatrix}
    1 & 0 & 0 \\
    0 & 1 & 0 \\
    0 & 0 & 1 \\
\end{pmatrix}
\]

\end{document}
