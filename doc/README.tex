\documentclass{article}
\usepackage[utf8]{inputenc}
\usepackage{parskip}
\usepackage{amsmath}
\usepackage{lmodern}

\title{\textbf{Mathall Matrices Documentation}}
\author{Filipondios}
\date{2022}

\begin{document}
\maketitle
\newpage

\textbf{Introduction}

In this document you will find the necessary documentation to understand the procedures and
thoughts in all Mathall operations with matrices.\\

You need to know that this isn't a math book and its not made by a mathematician, so the most
probable thing is that the explanations aren't the best or not exactly how a mathematician would
tell, because this document is made by a programmer just like every other out there.\\

\textbf{Greetings}

The only greetings i have to give is to my algebra book, that sometimes is unreadable but at least
opens its pages for me.

\newpage 

\section{Operations with Matrices}
\subsection{Matrices addition}

The very first function that was added to Mathall was the addition of two matrices. It was really
easy to perform, we only needed to add every element of each column and row with the other matrix.
There a restriction with the addition of two matrices: Both matrices must have the same
rows and columns. Knowing that, the resultant element of the row \textit{i} and column \textit{j} will be:

\[ e_{i_j} = m1_{i_j} + m2_{i_j} \]

Being $m_1$ the element in the row \textit{i} and column \textit{j} of a matrix called \textit{m1}
and $m_2$ the element in the row \textit{i} and column \textit{j} of a matrix called \textit{m2}.


Let's make an example and add two simple matrices:

\[
\begin{pmatrix}
    1 & 2 & 3 \\
    4 & 5 & 6 \\
    7 & 8 & 9 \\
\end{pmatrix}
  + 
\begin{pmatrix}
    9 & 8 & 7 \\
    6 & 5 & 4 \\
    3 & 2 & 1 \\
\end{pmatrix}
=
\begin{pmatrix}
    10 & 10 & 10 \\
    10 & 10 & 10 \\
    10 & 10 & 10 \\
\end{pmatrix}
\]

In terms of code, the most efficient way to do this operations is to making the
addition of each element in both matrices row by row. In \textit{Java} code:\\

for(int i = 0; i \textless resultant.length; i++)

\quad \quad \quad for(int j = 0; i \textless resultant.length; j++)

\quad \quad \quad \quad \quad \quad resultant[i][j] = matrix1[i][j] + matrix2[i][j];


\subsection{Matrices subtraction}

This is really the same as the matrix addition but instead of making the addition of every element,
the subtraction. Let's see an example:

\[
\begin{pmatrix}
    9 & 8 & 7 \\
    6 & 5 & 4 \\
    3 & 2 & 1 \\
\end{pmatrix}
  - 
\begin{pmatrix}
    8 & 8 & 7 \\
    6 & 4 & 4 \\
    3 & 2 & 0 \\
\end{pmatrix}
=
\begin{pmatrix}
    1 & 0 & 0 \\
    0 & 1 & 0 \\
    0 & 0 & 1 \\
\end{pmatrix}
\]

As you can imagine, the code is almost the same:\\

for(int i = 0; i \textless resultant.length; i++)

\quad \quad \quad for(int j = 0; i \textless resultant.length; j++)

\quad \quad \quad \quad \quad \quad resultant[i][j] = matrix1[i][j] - matrix2[i][j];\\


\subsection{Matrices product}

Matrix product is more complicated than the matrix addition and subtraction. As you have
seen, when doing the addition and subtraction of matrices, it generates a resulting matrix
with the same dimension ( dimension = Rows x Columns ) as both matrices that generate it.

When doing a matrix product, we get a matrix with a dimension made by the number of rows of
the first matrix and the number of columns of the second. There is a simple restriction: 
the number of columns of the first matrix must be the same as the second matrix number of rows.

The procedure to do the matrix product is the next: You have to multiply each element of 
each row of the first matrix with each element of each column of the second matrix and then
add all that values. The result of that addition will be stored in the position made by: 
the number of the row of the first matrix that you are multiplying and the column that you
are multiplying from the second.

Knowing that, lets see an example:

\[
\begin{pmatrix}
    1 & 2 & 3 \\
    4 & 5 & 6 \\
\end{pmatrix}
  * 
\begin{pmatrix}
    10 & 11 \\
    20 & 21 \\
    30 & 31 \\
\end{pmatrix}
=
\begin{pmatrix}
    1*10+2*20+3*30 & 1*11+2*21+3*31 \\
    4*10+5*20+6*30 & 4*11+5*21+6*31 \\
\end{pmatrix}
=
\begin{pmatrix}
    140 & 146 \\
    320 & 355 \\
\end{pmatrix}
\]

A classic example of matrix product can be done with a identity matrix, that won't modify
the resultant matrix.

\[
\begin{pmatrix}
    1 & 2 & 3 \\
    4 & 5 & 6 \\
    4 & 5 & 6 \\
\end{pmatrix}
  * 
\begin{pmatrix}
    1 & 0 & 0 \\
    0 & 1 & 0 \\
    0 & 0 & 1 \\
\end{pmatrix}
=
\begin{pmatrix}
    1 & 2 & 3 \\
    4 & 5 & 6 \\
    4 & 5 & 6 \\
\end{pmatrix}
\]

Lets traduce that in code:\\


for(int a = 0; a \textless matrix2[0].length; a++) \{

\quad \quad \quad for(int i = 0; i \textless matrix1.length; i++) \{

\quad \quad \quad \quad \quad \quad int resultant = 0;

\quad \quad \quad \quad \quad \quad for(int j = 0; j \textless matrix1[0].length; j++)

\quad \quad \quad \quad \quad \quad \quad \quad \quad element += element + matrix1[i][j]*matrix2[j][a];

\quad \quad \quad \quad \quad \quad resultant[i][a] = element;\\

\quad \quad \quad \}

\}

\end{document}
